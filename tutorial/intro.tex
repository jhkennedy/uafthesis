\chapter{Introduction}
\label{sec:intro_introduction}

How do correlations arise in a seemingly random complex system and
what are their signatures?  The study of self-organized criticality
(SOC) is one attempt to address this question.  SOC is a general
theory that has been applied in many physical and social fields to try
to help understand why very disparate systems share remarkably similar
quantitative signatures.  This dissertation studies three such
signatures of a SOC model, the running sandpile, and discusses what
they reveal about long time dynamical correlations in a SOC system and
how this can be applied to studies of confined and space plasmas.  The
signatures are the probability density function (PDF), the power
spectrum and the rescaled range ($R/S$).  The running sandpile has
been studied and used as a guide in SOC for over twenty years, since
shortly after the introduction of SOC itself in 1987.  This
dissertation overturns some of the conclusions and assumptions from
earlier studies that have been accepted since then and also presents
investigations of an extension of the model that has not been
previously studied.

\section{Randomness, Complex Systems and Long Time Correlations}
\label{sec:intro_rand-compl-syst}

The definition of randomness can be endlessly debated.  Starting with
quantum mechanics, one could say that the entire universe is random
and that predictability is impossible.  Even though people make
predictions that seem to be correct, the outcome of, for instance, a
chicken thrown through the air may actually be slightly off from
calculations due to the inherent randomness of the quantum world and
the lack of knowledge of all initial conditions of the system (the
universe).

%\bibliographystyle{unsrtabbrv3}
%\bibliography{thesis}
