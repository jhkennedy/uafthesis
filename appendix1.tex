\chapter{Do Your Socks Really Smell?}
\label{sec:comm-do-earthq}

Two really great papers are \cite{woodard04a,woodard04b}.  Another
good one is \cite{hurst51a}.

 \emph{X. Yang et al} present a detailed analysis of the
 first-return-time probability distribution (FRTDF) for earthquakes
 with magnitude equal to or larger than some prescribed threshold
 $M$~\cite{yang04a}. The data were extracted from the Southern
 California Seismographic Network (SCSN) catalog. Their conclusion is
 that the observed behavior fundamentally opposes what would be
 expected if the dynamics was governed by self-organized criticality
 (SOC). In this comment, we will however argue that the results
 reported in Ref.~\cite{yang04a}, far from discarding SOC for
 modelling earthquake dynamics, provide further evidence in favor of
 such a description when interpreted properly.
\begin{figure}
\centering
\includegraphics[width=8cm]{\figeps{debian-openlogo-nd}}
\caption{FRTDFs of the instantaneous avalanching activity in a
$L=2000$ sandpile for avalanches with sizes above different
thresholds (pdfs shifted for clarity). Inset: FRTDFs (in lin-log scale)
for same data after shuffling avalanches.}
\label{fig1}
\end{figure}

The opposite conclusion drawn by \emph{Yang et al} is due to a common
misconception about the nature of SOC temporal features. It is
contained in the sentence: ``One implication of earthquakes being SOC
is that an earthquake does not know how large it will become or, in
other words, the magnitude of an earthquake is completely random for a
given quake \ldots.'' (second page, first paragraph). Would this
statement hold, the test for SOC behavior they propose would be
adequate, since any measure of the temporal evolution of the system
activity should be invariant under shuffling or reordering of the
quakes in the sequence. \emph{Yang et al} perform this test on the
FRTDFs, finding them not invariant after the reordering. They
interpret this result correctly as a signature of strong temporal
correlations for quakes with magnitude larger than some minimum
threshold but then they claim that this is in contradiction with the
idea of SOC.

\bibliographystyle{unsrtabbrv3_TU}
\bibliography{thesis_TU}
